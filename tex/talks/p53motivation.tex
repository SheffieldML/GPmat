\begin{frame}
  \frametitle{Radiation Damage in the Cell}
  \begin{itemize}
  \item Radiation can damages molecules including DNA.
  \item Most DNA damage is quickly repaired---single strand breaks,
    backbone break.
  \item Double strand breaks are more serious---a complete disconnect along
    the chromosome.
  \item Cell cycle stages: 
    
    \begin{itemize}
    \item $\text{G}_{1}$: Cell is not dividing.
    \item $\text{G}_{2}$: Cell is preparing for meitosis, chromosomes have
      divided.
    \item S: Cell is undergoing meitosis (DNA synthesis).
    \end{itemize}
  \item Main problem is in $\text{G}_1$. In $\text{G}_2$ there are two
    copies of the chromosome.  In $\text{G}_1$ only one copy.
  \end{itemize}
\end{frame}


\begin{frame}
  \frametitle{p53 ``Guardian of the Cell''}
  \begin{itemize}
  \item Responsible for Repairing DNA damage
  \item Activates DNA Repair proteins
  \item Pauses the Cell Cycle (prevents replication of damage DNA)
  \item Initiates \emph{apoptosis} (cell death) in the case where damage can't
    be repaired.
  \item Large scale feeback loop with NF-$\kappa$B.
  \end{itemize}

\end{frame}


\begin{frame}
  \frametitle{p53 DNA Damage Repair}

  % 
  \begin{figure}
    \includegraphics[width=0.45\textwidth]{../../../sysbio/tex/diagrams/p53-unbound}\hfill\includegraphics[width=0.4\textwidth]{../../../sysbio/tex/diagrams/p53-bound}

    \caption{p53. \emph{Left} unbound, \emph{Right }bound to DNA. Images by David
      S. Goodsell from \protect\url{http://www.rcsb.org/} (see the``Molecule
      of the Month'' feature).}

  \end{figure}
\end{frame}


\begin{frame}
  \frametitle{p53}

  % 
  \begin{figure}
    \begin{centering}
      \includegraphics[angle=90,width=0.6\textwidth]{../../../sysbio/tex/diagrams/f013802.jpeg}
      \par\end{centering}

    \caption{Repair of DNA damage by p53. Image from \citet{Goodsell:p53tumor99}.}
  \end{figure}
\end{frame}


\begin{frame}
  \frametitle{Some p53 Targets}
  \begin{description}
  \item [{\emph{DDB2}}] DNA Damage Specific DNA Binding Protein 2. (also
    governed by C/ EBP-beta, E2F1, E2F3,...).
  \item [{\emph{p21}}] Cycline-dependent kinase inhibitor 1A (CDKN1A). A
    regulator of cell cycle progression. (also governed by SREBP-1a,
    Sp1, Sp3,... ).
  \item [{\emph{hPA26/SESN1}}] sestrin 1 Cell Cycle arrest.
  \item [{\emph{BIK}}] BCL2-interacting killer. Induces cell death (apoptosis)
  \item [{\emph{TNFRSF10b}}] tumor necrosis factor receptor superfamily,
    member 10b. A transducer of apoptosis signals.
  \end{description}
  
\end{frame}


\begin{frame}
  \frametitle{Modelling Assumption}
  \begin{itemize}
  \item Assume p53 affects targets as a single input module network motif
    (SIM).
  \end{itemize}
  % 
  \begin{figure}


    \begin{centering}
      \includegraphics[width=0.4\textwidth]{../../../sysbio/tex/diagrams/p53_sim}
    \end{centering}
    
    \caption{p53 SIM network motif as modelled by \citealt{Barenco:ranked06}.}
    
  \end{figure}
\end{frame}
