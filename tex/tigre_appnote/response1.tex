\documentclass{article}
\usepackage{csquotes}
\usepackage{amsmath}
\usepackage{amssymb}
\usepackage{url}

\newcommand{\tigre}{\emph{tigre}}
\newcommand{\myquote}[1]{\begin{quote}#1\end{quote}}

\setlength{\parindent}{0pt}
\addtolength{\parskip}{1.5mm}

\begin{document}

\begin{center}
\textbf{\LARGE Summary of changes and response to reviewer comments}
\end{center}

We wish to thank all the reviewers for their constructive comments
that have helped us further improve the paper and the package.

\section*{Reviewer 1}

%\begin{itemize}
%\item
  \begin{quote}
My only (minor) criticism is that I would have liked to see a slightly longer description (or maybe a figure?) to describe the main ideas behind the approach: inferring TFA and then ranking genes according to their match to the TFA (even though TFA is marginalised).
  \end{quote}

We have extended the description slightly by adding the following
sentence to the end of the third paragraph of Sec. 1:
``Candidate targets are then ranked by model
likelihood, which effectively measures how well the target fits a
model of regulation by the inferred TF.''

\clearpage

\section*{Reviewer 2}

%\begin{itemize}
%\item 
\myquote{Scope of the method: to infer the TF protein concentration, the TF itself needs to be known by specifying its identifier. This restriction should be spelled out clearly in the article, for example in the introduction. I have tried to omit the ''TF�� argument in the ''GPLearn�� command and it appears the first of the genes given in the ''targets�� list is taken to be the transcription factor. If it is possible to use a TF concentration determined through other means, then a worked example should be given in the documentation.}

It is possible to run the method without knowledge of the TF as
described.  This has been clarified in the documentation of the most
recent version of the package.

\myquote{Execution messages. The following message:
''Error in .gpsimUpdateKernels(model) : Singular chol(K) matrix!��
seems to appear when the GPLearn command is run. (joint model: $>$mt\_model$<$- \dots). It seems to be a warning rather than an error message, although when this occurs, execution time appears to be longer. This should be corrected in the package itself so that the end user does not abort the run.

The message  appearing after launching the GPLearn method is a bit too terse: the gene used as TF should be separate from the target genes.}

These have been corrected in most recent versions of the package for
R-2.12 and development versions.

\clearpage

\section*{Reviewer 3}

%\begin{itemize}
%\item 
\myquote{I have attempted to install the package in order to test it, however, the installation failed (on MAC OS X 10.6), with a message suggesting that "tigre" was not available on the bioconductor webpage: 

source("http://www.bioconductor.org/biocLite.R")
biocLite("tigre")
Using R version 2.10.0, biocinstall version 2.5.11.
Installing Bioconductor version 2.5 packages:
[1] "tigre"
Please wait...

Warning message:
In getDependencies(pkgs, dependencies, available, lib) :
 package 'tigre' is not available}

As noted in the abstract, tigre is included in Bioconductor since
release 2.6 for R 2.11. Unfortunately Bioconductor does not allow us
to support earlier R releases.

\myquote{The authors have also submitted a package called "tigreBrowser" as another application note to Bioinformatics. The purpose of tigreBrowser is to "handle the results data produced by tigre" (and in general, similar outputs from similar methods), which makes me consider whether it wouldn't be more reasonable to submit both packages as a single application note.}

Due to different nature of the packages and mostly disjoint set of
authors, we have decided not to combine the application notes.

\myquote{The method is parallelisable, however paralellisation has not automatically been implemented and the authors suggest the users to submit several individual jobs in parallel. I suggest revising the section "2.1 Parallelisation" due to the following reasons: (i) It is not clear from the manuscript why the authors prefer manual paralellisation to parallelisation using MapReduce, Hadoop, RHIPE. (ii) It feels like the parallelization mentioned in the abstract sets the expectations too high.}

Sec. 2.1 has been clarified to explain that we prefer manual
parallelisation because of easier software requirements:
``\emph{In terms of required additional software,}
the easiest way to run \tigre{} in parallel is to simply split the...''

The ending of the abstract has been edited to hopefully avoid inflated
expectations:
``\emph{The method has been
designed with efficient parallel implementation in mind, and the
package supports parallel operation even without additional
software.}''

\myquote{I would suggest the style of writing is at times too informal/general, and should be revised. A few examples: "is relatively fast, but still takes some time" (pg. 1, line 58, column 2), "allows relatively easy running" (pg 2, line 22), "tigre can provide useful results" (pg 2, line 28), "We are working on a method based on a more realistic model, however it will be challenging to turn that framework in to something as convenient as tigre." (pg. 2, line 40).}

The highlighted passages have been reworded:
``The method implemented by \tigre{} includes no Monte Carlo simulations,
but typical running times still range from
seconds to up to a few minutes per gene depending on the data and the
number of targets in the models.''

``This approach provides an alternative method
of running the ranking in a highly parallelised fashion in a cloud
computing setting.''

``We are working on a method based on a more realistic
non-linear model.  Such models require, however, more advanced
computational techniques that often tend to be more fragile.
\tigre{} avoids this pitfall by using a simple enough model allowing
very robust inference while effectively capturing the
essential degrees of freedom in the transcription regulatory process.''

\myquote{pg. 1, line 53, column 2: I suggest rewriting the sentence "Fitted models can be stored very compactly by just storing their parameters." I thought this would be a standard way of storing a model. It is not clear what is the alternative, less compact way of storing the models.}

Revised to:
``Depending on the number of genes
in the model, the model is defined using from 7 up to a few dozen
parameters, allowing very compact storage of fitted models.''

\myquote{It would be useful to add Links to the webpage and the code, and mention that the User Guide is available.}

Revised Availability section of the abstract to mention this:
``The \tigre{} package is included in Bioconductor since release 2.6 for
R 2.11. The package and users' guide are available at
\url{http://www.bioconductor.org}.''

%\end{itemize}

\end{document}
