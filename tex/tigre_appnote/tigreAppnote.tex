\documentclass{bioinfo}
\copyrightyear{2010}
\pubyear{2010}

\newcommand{\tigre}{\emph{tigre}}

\begin{document}
\firstpage{1}

\title[tigre]{tigre: Transcription factor Inference through Gaussian process Reconstruction of Expression for Bioconductor}
\author[Honkela \textit{et~al.}]{Antti Honkela\,$^{1,2,*}$, Pei Gao\,$^{3}$, Jonatan Ropponen\,$^{1}$, Magnus Rattray\,$^{4,*}$ and Neil D.\ Lawrence\,$^{4,}$\footnote{to whom correspondence should be addressed}}
\address{$^{1}$Department of Information and Computer Science, Aalto
  University, Helsinki, Finland\\
  $^{2}$Helsinki Institute for Information Technology HIIT,
  University of Helsinki, Helsinki, Finland\\
  $^{3}$Department of of Public Health and Primary Care, University of
  Cambridge, Cambridge, UK\\
  $^{4}$ Sheffield Institute for Translational Neuroscience and
  Department of Computer Science, University of Sheffield, Sheffield, UK}

\history{Received on XXXXX; revised on XXXXX; accepted on XXXXX}

\editor{Associate Editor: XXXXXXX}

\maketitle

\begin{abstract}

\section{Summary:}
\tigre{} is an R/Bioconductor package for inference of transcription
factor activity and ranking candidate target genes from gene
expression time series.  The underlying methodology is based on
Gaussian process inference on a differential equation model that
allows using short unevenly sampled time series.  The method has been
designed with efficient parallel implementation in mind, and the
package supports parallel operation even without additional
software.

\section{Availability:}
The \tigre{} package is included in Bioconductor since release 2.6 for
R 2.11. The package and users' guide are available at
\href{http://www.bioconductor.org}{http://www.bioconductor.org}.

\section{Contact:} \href{antti.honkela@hiit.fi}{antti.honkela@hiit.fi},
\href{m.rattray@sheffield.ac.uk}{m.rattray@sheffield.ac.uk},\\
\href{n.lawrence@sheffield.ac.uk}{n.lawrence@sheffield.ac.uk}
\end{abstract}

\section{Introduction}

Understanding genome function through reverse engineering of gene
regulatory relationships from experimental data is one of the key
challenges in current biology~\citep{Bansal2007a,Bickel2009e}.  One
popular technique is to use gene expression time series to infer these
relationships.  Unfortunately most real world expression time series
are short~\citep{Ernst2005} and contain insufficient information for
any realistic reconstruction of the gene regulatory
network~\citep{Smet2010}.

Recognising this, the \tigre{} package aims at answering a much
simpler question: given time series expression data where a
transcription factor (TF) is changing its activity, which genes are
plausibly regulated by that TF.  As a result, it provides a ranking of
tested target genes according to their likelihood of being targets of
the TF.

The underlying methodology was presented by~\citet{Honkela2010PNAS},
who showed that it can yield remarkably accurate predictions from very
limited data, often attaining more accurate results based on the
simple wild time series expression data than could be obtained using
TF knock-out data.  The method is based on a linear ordinary
differential equation model of TF protein translation and
transcriptional regulation.  Placing a non-parametric Gaussian process
prior on unobserved mRNA concentration or protein activity time courses
leads to a
joint Gaussian process model over the protein activity level and all
gene expression levels~\citep{Gao2008}.  The Gaussian processes can be
marginalised out analytically while the model parameters are optimised
using maximum likelihood.  Candidate targets are then ranked by model
likelihood, which effectively measures how well the target fits a
model of regulation by the inferred TF.

The method was originally implemented in MATLAB,
making handling of genomic data sets much more cumbersome than in the
new Bioconductor implementation, which benefits from all the
pre-processing, annotation and other tools in
Bioconductor~\citep{Gentleman2004}.

% \begin{methods}
% \section{Methods}

% The \tigre{} package is an implementation of the Gaussian process
% single input motif framework of~\citet{Gao2008} and the related TF
% target ranking method of~\citet{Honkela2010PNAS}.  This framework is
% based on a linear ordinary differential equation model of TF protein
% translation and transcription regulation described by the equations
% \begin{align}
%   \frac{\mathrm{d}p(t)}{\mathrm{d}t} & = f(t) - \delta
%   p(t) \ , \label{eq:translation_ode} \\
%   \frac{\mathrm{d}m_j(t)}{\mathrm{d}t} & = B_j+S_j p(t)-D_j m_j(t) \ , \label{eq:transcription_ode}
% \end{align}
% where $p(t)$ is the TF protein at time $t$, $m_j(t)$ is the $j$th target mRNA
% concentration and $f(t)$ is the TF mRNA. The parameters $B_j$, $S_j$ and $D_j$ are the
% baseline transcription rate, sensitivity and decay rate respectively
% for the mRNA of the $j$th target \citep[as described by][]{Barenco2006a}.
% The parameter $\delta$ is the decay rate of the TF
% protein~\citep{Honkela2010PNAS}. 

% Placing a Gaussian process prior on $f(t)$\footnote{If the TF protein
%   is under significant post-translational regulation,
%   Eq.~(\ref{eq:translation_ode}) may be omitted and the prior placed
%   directly on $p(t)$.  In this case multiple known targets are needed
%   to reliably infer $p(t)$.} leads to a joint Gaussian process over
% the three continuous-time functions $f(t),p(t)$ and $m_j(t)$.  Data are observations of the expression levels at arbitrary specific times (not necessarily evenly spaced) and we assume a Gaussian noise model: $\hat{m}_j(t_i) \sim N(m_j(t_i),\sigma_{i,m_j}^2)$ and $\hat{f}(t_i) \sim N(f(t_i),\sigma_{i,f}^2)$ with known (derived from \emph{puma}, see below) or estimated gene-specific noise variance parameters. The parameters of the model as well as other parameters of the Gaussian process covariance are
% optimised by maximising the marginal likelihood.
% \end{methods}

\section{Implementation}

The \tigre{} package is tightly integrated into the Bioconductor
microarray data analysis framework, especially with the \emph{puma}
package~\citep{Pearson2009} which associates expression levels derived from
high-density microarray data with Bayesian confidence regions
analogous to standard errors.
Expression data from other platforms and analysis methods that do not
provide standard error estimates can also be easily used.  In this
case the observation noise level is estimated as part of fitting the
model.

Functions are provided for processing data to a format suitable for
the method, including estimation of standard errors of unlogged expression
values for \emph{puma} processed data (currently \emph{puma} standard errors are for logged expression); fitting the models individually
or in a batch; and plotting the models to assess the fit.  The models
are fitted using scaled conjugate gradient
optimisation~\citep{Moller:scg93}.  Depending on the number of genes
in the model, the model is defined using from 7 up to a few dozen
parameters, allowing very compact storage of fitted models.

\subsection{Parallelisation of the Ranking}

The method implemented by \tigre{} includes no Monte Carlo simulations,
but typical running times still range from
seconds to up to a few minutes per gene depending on the data and the
number of targets in the models.
Therefore \tigre{} has been designed for efficient parallelisation.  In the
ranking, each gene is handled completely independently.  This makes
the code trivially parallelisable up to the level of running each gene
in a separate machine.  This linear parallelisation to potentially
several thousands of processes is impossible in more tightly coupled
modelling methods.

In terms of required additional software,
the easiest way to run \tigre{} in parallel is to simply split the
task to a number of jobs that can be run independently, possibly by
submitting them as independent jobs to a queuing system.  The package
does not include integration with an MPI environment, because that
matches its requirements poorly.  A number of independent jobs should
also be easier to schedule than a single large MPI job.

An alternative technique for running \tigre{} in parallel is based on
MapReduce~\citep{Dean2008}.  The ranking approach fits this paradigm
perfectly: the mapper fits models to each gene independently and the
reducer forms the final ranking.  We have implemented this approach
using Hadoop and RHIPE.  This approach provides an alternative method
of running the ranking in a highly parallelised fashion in a cloud
computing setting.

\section{Discussion}

\tigre{} can fit the models already on very short time series.  We
have successfully applied it to data sets with as few as 6 and 7 time
points~\citep{Honkela2010PNAS,Honkela2010MLSP}.

The output of \tigre{} is a ranked set of models for each candidate
target gene together with their associated log-likelihoods used to
form the ranking.  Direct interpretation of the log-likelihood scores
can be difficult.  We have found it very useful to use visualisations
of the models to assess the fit and to select thresholds for filtering
weakly expressed genes \citep[for details, see][]{Honkela2010PNAS}.
We have also developed a web interface called \emph{tigreBrowser}
(\href{http://pypi.python.org/pypi/tigreBrowser}{http://pypi.python.org/pypi/tigreBrowser}) to
aid in browsing the visualisations.

The linearity of the differential equation transcription model greatly
simplifies the algorithm, but it may be too crude an assumption for some
situations.  We are working on a method based on a more realistic
non-linear model.  Such models require, however, more advanced
computational techniques that often tend to be more fragile.
\tigre{} avoids this pitfall by using a simple enough model allowing
very robust inference while effectively capturing the
essential degrees of freedom in the transcription regulatory process.

\section*{Acknowledgement}

The authors wish to thank Jennifer Withers for useful comments on the
package and Miika-Petteri Matikainen for implementing \emph{tigreBrowser}.

\paragraph{Funding\textcolon}
This work was supported by the Academy of Finland [121179 to A.H.];
the Engineering and Physical Sciences Research Council [EP/F005687/1
to M.R. and N.D.L.]; and
the IST Programme of the European Community, under the PASCAL2
Network of Excellence [IST-2007-216886].
This publication only reflects the authors' views.

\small

\bibliographystyle{myabbrvnat}
\bibliography{tigre}

\end{document}
